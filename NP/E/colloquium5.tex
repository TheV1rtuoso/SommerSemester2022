

           \documentclass[12pt]{article}
           \usepackage{fullpage,enumitem,amsmath,amssymb,graphicx}
           
           \title{Colloquium 5 - Concurrent Programming}
           \author{Simon Hasir -- 7006072}
           
           \usepackage{listings}
           
           \begin{document}
             \maketitle
             \noindent
             \rule{\linewidth}{0.4pt}
             
           \section{TE-1}
           \begin{enumerate}[label=(\alph*)]
               \item No, it's a destructive channel, only one reciever but inititially it's \textit{"send} to all 
               \item  No, it runs the first case with a matching condition and chooses the defaults non deterministically
               \item  No, intchan1 is asynchronous and doesn't wait for a receiver whereas intchan only continues with execution when the massage is fetched by another agent.
               \item  No, when every agent terminates
               \item  Yes, if you fetch from a sync channel without any input
           \end{enumerate}
           \section{TE-2}
           \begin{enumerate}[label=(\alph*)]
               \item return; oder mainAgent(){};
               \item void x() \{...\}
               \item \begin{lstlisting}
mainAgent {
        select {
            default: {
                P
            }
            default: {
                P
            }
        }
}
                    \end{lstlisting} 
                \item $\alpha;P;$
                \item \begin{lstlisting}
mainAgent {
        start(P());
        start(Q());
}
                \end{lstlisting} 
            \end{enumerate}
    \section{TE-3}
    \begin{enumerate}[label=(\alph*)]
        \item 
    \begin{lstlisting}
    mainAgent {
    stringchan toastchan;
    boolchan ready, go;

    // A toaster with two slots..
    start(slot(toastchan, ready, go));
    start(slot(toastchan, ready, go));
    // ..and two users
    start(user("Conny", toastchan));
    start(user("Dieter", toastchan));
    
    for(int i = 0; i < 2; i++){
        <? ready;
    }
    for(int i = 0; i < 2; i++){
        go <! true;
    }

}

void slot(stringchan toastchan, boolchan ready, boolchan go) {
    ready <! true;
    <? go; // wait for go before toasting
    string toast = <? toastchan;
    toastchan <! "Crispy " + toast;
}
    \end{lstlisting}
    \item After l. 24, l. 25 can be run instead of slots(). Solution: another stringchan. toast = <? toastchan2 l.19 toastchan2 <! 
    \end{enumerate}
    \newpage
    \section{TE-4}
    \begin{enumerate}
        \item fibonacci sequence
        \item \begin{lstlisting}
while (run) {
    select {
        case c <! v: {}
        case u = <? c: {
            int i = <? k;
            if (i == n) { //terminate and send term signal
                run = false;
                b <! true;
                println(v);
            }
            else {
                k <! i+1; //inc counter
                v = v + u;
                c <! v;    
            }  
        }
        case <?b: {
            run = false;
        }
    };
}
        \end{lstlisting}
    \end{enumerate}
\end{document}




\documentclass[12pt]{article}
\usepackage{fullpage,enumitem,amsmath,amssymb,graphicx, hyperref}

\title{Colloquium 5 - Concurrent Programming}
\author{Simon Hasir -- 7006072}

\usepackage{listings}
           
\begin{document}
    \maketitle
    \noindent
    \rule{\linewidth}{0.4pt}
    
    \section{TF-1}
    \begin{enumerate}[label=(\alph*)]
        \item False, you can define recursive functions and call functions of the same monitor
        without getting a new lock.
        \item True, but the deadlock problem is not solved.
        \item False, interleaving its still possible.
        \item Yes. With locks at the begining and end.
    \end{enumerate}
    \section{TF-2}
    \begin{enumerate}[label=(\alph*)]
        \item Datarace if line 14 is executed before l.6.\\
        Fix: Use the same lock for accesses to b.
        \item Deadlock, if mainagent gets lock m2 and a1 gets lock m1. \\
        Fix: Use the locks in the same order everytime.
        \item Valid, due to purely sequential execution.
    \end{enumerate}
    \section{TF-3}
    \href{https://pseuco.com/#/edit/remote/qsdwbcrf6c970rdrtnmq}{See implementation}
    
\end{document}


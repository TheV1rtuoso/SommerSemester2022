
\documentclass[12pt]{article}
\usepackage{fullpage,enumitem,amsmath,amssymb,graphicx}
\usepackage{wasysym}
\title{System Security Assignment 1}
\author{Simon Hasir -- 7006072}

\begin{document}
    \maketitle
    \section*{Aufgabe 1}
    \begin{enumerate}[label=(\alph*)]
    \item \begin{enumerate}[label=(\roman*)]
        \item SELECT DISTINCT Geburtsjahr \\
        FROM Personen JOIN Schüler*Innen \\
        ON Wohnort='Saarland' AND Schulform = 'Grundschule' AND PID=SID;
        \item SELECT DISTINCT Dauer\\
        FROM korregieren AS a JOIN Klausuren AS b\\
         ON a.Klausur = b.KID AND b.Fach = 'Französisch'\\
        JOIN unterichten AS c ON b.LehrerIn = c.LehrerIn \\
        WHERE c.Wohnort = 'Saarbrücken' and c.Schulform = 'Grundschule'
        \item SELECT name, min(note) \\
        FROM Personen\\
        JOIN SchülerIn ON PID=SID \\
        JOIN korregieren ON  SID=SchülerIn\\
        GROUP BY SID, name HAVING avg(note) $\le$ 3 
    \end{enumerate} % Task a
    \item \begin{enumerate}[label=(\roman*)]
        \item $R1$ := Schüler*Innen $\Bowtie _{SID= Schueler*Innen}$ unterichten\\
        $\sigma_{Fach='Geschichte' \land Klassenraum=202} (R1)$
        \item $R1$ := korregieren $\Bowtie_{LehrerIn=LID}$ Lehrer*Innen\\
        $R2$ := $\sigma_{Hauptfach='Physik' \land Gehalt>= 2000}$\\
        $\pi$\textsubscript{MIN(Note)}($\sigma$\textsubscript{MAX(Note) $<$ 5}($\gamma$\textsubscript{LID, MIN(note), MAX(note)}(R2)))
    \end{enumerate}
    \end{enumerate}
    \section*{Aufgabe 2}
    \begin{enumerate}[label=(\alph*)]
        \item \begin{enumerate}[label=(\roman*)]
        \item SELECT SID, count(*)\\
        Schüler*Innen AS s JOIN unterichten AS u 
        \\ON s.klasenstufe = 5 \\
        GROUP BY SID 
        HAVING u.Uhrzeit = 14 AND u.Fach = 'Erdkunde'
        % What happens with Zeros
        \item SELECT SID, Klassenstufe \\
        FROM Schüler as s JOIN korregieren as k ON s.SID = k.Schüler*Innen
        GROUP BY SID having count(*) = 5
        \item SELECT SID, avg(Gehalt) FROM \\
        (SELECT avg(Gehalt) FROM Lehrer l JOIN unterichten u on u.LehrerIn = l.LID WHERE Fach=Sport 
        \\GRUOP BY l.LID, u.SchuelerIn having count(*) = 4)
        JOIN  
        \end{enumerate}
        \item \begin{enumerate}[label=(\roman*)]
            \item Geburtsjahr der Personen die auf dem Gymnasium sind und deren Wohnort mit 'brücken' aufhört, 
            wobei die Jahreszahl einmalig ist. Die Geburtsjahr sind absteigendnach der Klassenstufe der Schueler sotiert einmalig ist. Die Geburtsjahr sind absteigend 
            nach der Klassenstufe der Schueler sotiert,
            \item Die Lehrer und das Datum des ersten unterichts, 
            welche das Fach 'Musik' unterichten und noch nie eine 6 gegeben hat.
        \end{enumerate}

    \end{enumerate}
    \section*{Aufgabe 3}
    \begin{enumerate}
        \item HAVING ohne GROUPBY
        \item WHERE hinter GROUPBY
        \item Umbennenung nach Accessing 
    \end{enumerate}


\end{document}
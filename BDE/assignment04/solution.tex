
\documentclass[12pt]{article}
\usepackage{fullpage,enumitem,amsmath,amssymb,graphicx}
\usepackage{wasysym}
\title{System Security Assignment 1}
\author{Simon Hasir -- 7006072}

\begin{document}
    \maketitle
    \section*{Aufgabe 1}
    \begin{enumerate}[label=(\alph*)]
    \item \begin{enumerate}[label=(\roman*)]
        \item SELECT Geburtsjahr \\
        FROM Personen JOIN Schüler*Innen \\
        ON Wohnort='Saarland' AND Schulform = 'Grundschule' AND PID=SID;
        \item SELECT Gehalt\\
        FROM korregieren AS a JOIN Klausuren AS b\\
         ON a.Klausur = b.KID AND b.Fach = 'Französisch'\\
        JOIN unterichten AS c ON b.LehrerIn = c.LehrerIn \\
        WHERE c.Wohnort = 'Saarbrücken' and c.Schulform = 'Grundschule'
        \item SELECT name, min(note) \\
        FROM Personen\\
        JOIN SchülerIn ON PID=SID \\
        JOIN korregieren ON  SID=SchülerIn\\
        GROUP BY SID, name HAVING avg(note) $\le$ 3 
    \end{enumerate} % Task a
    \item \begin{enumerate}[label=(\roman*)]
        \item $R1$ := Schüler*Innen $\Bowtie _{SID= Schueler*Innen}$ unterichten\\
        $\sigma_{Fach='Geschichte' \land Klassenraum=202} (R1)$
        \item $R1$ := korregieren $\Bowtie_{LehrerIn=LID}$ Lehrer*Innen\\
        $R2$ := $\sigma_{Hauptfach='Physik' \land Gehalt>= 2000}$\\
        $\pi$\textsubscript{MIN(Note)}($\sigma$\textsubscript{MAX(Note) $<$ 5}($\gamma$\textsubscript{LID, MIN(note), MAX(note)}))
    \end{enumerate}
    \end{enumerate}
    \section*{Aufgabe 2}
    \begin{enumerate}[label=(\alph*)]
        \item SELECT SID, count(*)\\
        Schüler*Innen AS s JOIN unterichten AS u 
        \\ON s.klasenstufe = 5 WHERE  TIME(u.Datum) = 14 AND u.Fach = 'Erdkunde'\\
        GROUP BY SID 
        % What happens with Zeros
        \item SELECT DISTINCT Klassenstufe \\
        FROM Schüler as s JOIN korregieren as k ON s.SID = k.Schüler*Innen
        GROUP BY SID having count(*) = 5
    \end{enumerate}

\end{document}